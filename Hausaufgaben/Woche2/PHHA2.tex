% Options for packages loaded elsewhere
\PassOptionsToPackage{unicode}{hyperref}
\PassOptionsToPackage{hyphens}{url}
%
\documentclass[
]{article}
\usepackage{amsmath,amssymb}
\usepackage{lmodern}
\usepackage{iftex}
\ifPDFTeX
  \usepackage[T1]{fontenc}
  \usepackage[utf8]{inputenc}
  \usepackage{textcomp} % provide euro and other symbols
\else % if luatex or xetex
  \usepackage{unicode-math}
  \defaultfontfeatures{Scale=MatchLowercase}
  \defaultfontfeatures[\rmfamily]{Ligatures=TeX,Scale=1}
\fi
% Use upquote if available, for straight quotes in verbatim environments
\IfFileExists{upquote.sty}{\usepackage{upquote}}{}
\IfFileExists{microtype.sty}{% use microtype if available
  \usepackage[]{microtype}
  \UseMicrotypeSet[protrusion]{basicmath} % disable protrusion for tt fonts
}{}
\makeatletter
\@ifundefined{KOMAClassName}{% if non-KOMA class
  \IfFileExists{parskip.sty}{%
    \usepackage{parskip}
  }{% else
    \setlength{\parindent}{0pt}
    \setlength{\parskip}{6pt plus 2pt minus 1pt}}
}{% if KOMA class
  \KOMAoptions{parskip=half}}
\makeatother
\usepackage{xcolor}
\usepackage[margin=1in]{geometry}
\usepackage{graphicx}
\makeatletter
\def\maxwidth{\ifdim\Gin@nat@width>\linewidth\linewidth\else\Gin@nat@width\fi}
\def\maxheight{\ifdim\Gin@nat@height>\textheight\textheight\else\Gin@nat@height\fi}
\makeatother
% Scale images if necessary, so that they will not overflow the page
% margins by default, and it is still possible to overwrite the defaults
% using explicit options in \includegraphics[width, height, ...]{}
\setkeys{Gin}{width=\maxwidth,height=\maxheight,keepaspectratio}
% Set default figure placement to htbp
\makeatletter
\def\fps@figure{htbp}
\makeatother
\setlength{\emergencystretch}{3em} % prevent overfull lines
\providecommand{\tightlist}{%
  \setlength{\itemsep}{0pt}\setlength{\parskip}{0pt}}
\setcounter{secnumdepth}{-\maxdimen} % remove section numbering
\ifLuaTeX
  \usepackage{selnolig}  % disable illegal ligatures
\fi
\IfFileExists{bookmark.sty}{\usepackage{bookmark}}{\usepackage{hyperref}}
\IfFileExists{xurl.sty}{\usepackage{xurl}}{} % add URL line breaks if available
\urlstyle{same} % disable monospaced font for URLs
\hypersetup{
  pdftitle={Hausaufgabe 1},
  pdfauthor={Team 4 (Milena Mensching, Justus Weyers)},
  hidelinks,
  pdfcreator={LaTeX via pandoc}}

\title{Hausaufgabe 1}
\author{Team 4 (Milena Mensching, Justus Weyers)}
\date{2022-11-14}

\begin{document}
\maketitle

\hypertarget{aufgabe-1}{%
\section{Aufgabe 1}\label{aufgabe-1}}

\textbf{Sie wiegen mit einer digitalen Waage, deren kleinste
Schrittweite (d.h. Auflösung) 0,1 g beträgt, einen Apfel. Der Apfel
wiegt laut Anzeige 120,0 g. Stellen Sie sich vor, nach der Messung
stellen Sie fest, dass der Tisch auf dem die Waage steht, schief steht.
Die Schieflage beträgt ca. 1-2 Grad.}

\begin{itemize}
\tightlist
\item
  \textbf{Berechnen Sie welche systematische Abweichung dadurch
  entsteht.}
\item
  \textbf{Ist es richtig anzunehmen, dass die Messung dennoch korrekt
  war und die Messung aufgrund der Messunsicherheiten als richtig
  angenommen werden kann?}
\end{itemize}

\textbf{geg.:} Bestwert Masse \(m = 120,0\); Geräteart: digital;
Auflösung \(a=0,1g\). Als Winkel \(\alpha\) wird der Extremfall gewählt,
bei dem die Neigung 2° beträgt.

\textbf{ges.:} Systematische Abweichung \(e\)

\textbf{Rechnung:}

Bei einer schiefen Ebene nimmt der Betrag der Normalkraft um den Kosinus
des Neigunswinkels ab:
\[e = m*(1-cos(\alpha)) = 120,0g*(1-cos(2)) = 0,073g\]

Die Messunsicherheit beträgt:
\[ u = \frac{a}{2\sqrt{3}} \Rightarrow \frac{0,1g}{2\sqrt{3}} = 0,029g\]
Damit ist die Systematische Abweichung größer, als die Messunsicherheit.
Das Ergebnis ist also im Rahmen der Messunsicherheit nicht korrekt.

\hypertarget{aufgabe-2}{%
\section{Aufgabe 2:}\label{aufgabe-2}}

\textbf{Es wurden 3 verschiedene Techniken genutzt, um den Durchmesser
und Umfang eines runden Objektes zu messen und daraus experimentell die
Größe \(\pi\) zu bestimmen. Methode 1 ergibt: 3,133 ± 0,007, Methode 2
3,1609 ± 0,0002 und Methode 3 war 3,14 ± 0,03. Welche der 3 Methoden ist
die präziseste? Begründen Sie Ihre Antwort.}

Methode 2 liefert das präziseste Ergebnis. Dieses wurde hier auf vier
Nachkommastellen genau angegeben. Die hohe Präzision sagt allerdings
nichts über die Richtigkeit des Ergebnisses aus.

\hypertarget{aufgabe-4}{%
\section{Aufgabe 4}\label{aufgabe-4}}

\textbf{Tom hängt drei Gewichte an eine Feder. Er benutzt eine Waage
(die Messunsicherheit des Gerätes \(u_{Waage}\) beträgt 0,05g, die
Auflösung der Waage beträgt 0,1g) um die drei Massen einmalig zu
bestimmen. Tom liest auf der Waage folgenden Werte für die Massen ab:}

\begin{itemize}
\tightlist
\item
  \(M_1= 30,2 g\)
\item
  \(M_2= 9,8 g\)
\item
  \(M_3= 5,1 g\)
\end{itemize}

\textbf{Wie schwer ist die gesamte Masse? Vergessen Sie nicht die
Messunsicherheit dazu.}

Berechnung der Skalen-Unsicherheit \(u_{Skala}\) mit \(a=0,1g\):
\[u_{Skala} = \frac{a}{2\sqrt{6}} = \frac{0,1g}{2\sqrt{6}}\] Berechnung
der Geräteunsicherheit \(u_{Gerät}\):
\[u_{Gerät} =\pm \sqrt{u_{Skala}^2+u_{Waage}^2} =\pm \sqrt{(\frac{0,1}{2\sqrt{6}})^2+(0,05)^2}g = \pm 0,054g\]

Formel für die Berechnung der Gesamtmasse \(m_{ges}\):
\[m_{ges} = m_1+m_2+m_3\]

Berechnung der Messunsicherheit \(u\): \begin{align*}
\begin{split}
u&=\pm \sqrt{(\frac{\partial m_{ges}}{\partial m_1}*u_{Gerät})^2+(\frac{\partial m_{ges}}{\partial m_2}*u_{Gerät})^2+(\frac{\partial m_{ges}}{\partial m_3}*u_{Gerät})^2}\\
 &= \sqrt{((1+m_2+m_3)*u_{Gerät})^2+((m_1+1+m_3)*u_{Gerät})^2+((m_1+m_2+1)*u_{Gerät})^2}\\
\Rightarrow &= \sqrt{((1+9,8+5,1)*0,054)^2+((30,2+1+5,1)*0,054)^2+((30,2+9,8+1)*0,054)^2}\\
&= 3,1g\\
\end{split}
\end{align*}

Die Gesamtmasse \(m_{ges}\) beträgt also:
\[m_{ges}=((30,2+9,8+5,1)\pm 3,1)g = (45,1 \pm 3,1)g\] \# Aufgabe 6
\textbf{Jo will die Periodendauer T eines Pendels experimentell
bestimmen. Jo entscheidet statt einer einzelnen Periode 20 Perioden mit
der Stoppuhr zu messen. Wie kann Jo die Periodendauer T und ihre
Messunsicherheit bestimmen? Wenn Sie jetzt die Ergebnisse dieser Übung
reflektieren, würden Sie sagen dass es um die Periodendauern eines
Pendels zu bestimmen, besser ist mehrere Perioden zu messen oder
wenige?}

Für die mittlere Periodendauer lässt sich folgende Formel aufstellen:
\[T = \frac{t}{n}\]

\begin{itemize}
\tightlist
\item
  \(T\): Gemittelte Periodendauer
\item
  \(t\): Gemessene Zeit
\item
  \(n\): Anzahl von Perioden
\end{itemize}

Die Messunsicherheit berechnetsich als:

\end{document}

% Options for packages loaded elsewhere
\PassOptionsToPackage{unicode}{hyperref}
\PassOptionsToPackage{hyphens}{url}
%
\documentclass[
]{article}
\usepackage{amsmath,amssymb}
\usepackage{lmodern}
\usepackage{iftex}
\ifPDFTeX
  \usepackage[T1]{fontenc}
  \usepackage[utf8]{inputenc}
  \usepackage{textcomp} % provide euro and other symbols
\else % if luatex or xetex
  \usepackage{unicode-math}
  \defaultfontfeatures{Scale=MatchLowercase}
  \defaultfontfeatures[\rmfamily]{Ligatures=TeX,Scale=1}
\fi
% Use upquote if available, for straight quotes in verbatim environments
\IfFileExists{upquote.sty}{\usepackage{upquote}}{}
\IfFileExists{microtype.sty}{% use microtype if available
  \usepackage[]{microtype}
  \UseMicrotypeSet[protrusion]{basicmath} % disable protrusion for tt fonts
}{}
\makeatletter
\@ifundefined{KOMAClassName}{% if non-KOMA class
  \IfFileExists{parskip.sty}{%
    \usepackage{parskip}
  }{% else
    \setlength{\parindent}{0pt}
    \setlength{\parskip}{6pt plus 2pt minus 1pt}}
}{% if KOMA class
  \KOMAoptions{parskip=half}}
\makeatother
\usepackage{xcolor}
\usepackage[margin=1in]{geometry}
\usepackage{color}
\usepackage{fancyvrb}
\newcommand{\VerbBar}{|}
\newcommand{\VERB}{\Verb[commandchars=\\\{\}]}
\DefineVerbatimEnvironment{Highlighting}{Verbatim}{commandchars=\\\{\}}
% Add ',fontsize=\small' for more characters per line
\usepackage{framed}
\definecolor{shadecolor}{RGB}{248,248,248}
\newenvironment{Shaded}{\begin{snugshade}}{\end{snugshade}}
\newcommand{\AlertTok}[1]{\textcolor[rgb]{0.94,0.16,0.16}{#1}}
\newcommand{\AnnotationTok}[1]{\textcolor[rgb]{0.56,0.35,0.01}{\textbf{\textit{#1}}}}
\newcommand{\AttributeTok}[1]{\textcolor[rgb]{0.77,0.63,0.00}{#1}}
\newcommand{\BaseNTok}[1]{\textcolor[rgb]{0.00,0.00,0.81}{#1}}
\newcommand{\BuiltInTok}[1]{#1}
\newcommand{\CharTok}[1]{\textcolor[rgb]{0.31,0.60,0.02}{#1}}
\newcommand{\CommentTok}[1]{\textcolor[rgb]{0.56,0.35,0.01}{\textit{#1}}}
\newcommand{\CommentVarTok}[1]{\textcolor[rgb]{0.56,0.35,0.01}{\textbf{\textit{#1}}}}
\newcommand{\ConstantTok}[1]{\textcolor[rgb]{0.00,0.00,0.00}{#1}}
\newcommand{\ControlFlowTok}[1]{\textcolor[rgb]{0.13,0.29,0.53}{\textbf{#1}}}
\newcommand{\DataTypeTok}[1]{\textcolor[rgb]{0.13,0.29,0.53}{#1}}
\newcommand{\DecValTok}[1]{\textcolor[rgb]{0.00,0.00,0.81}{#1}}
\newcommand{\DocumentationTok}[1]{\textcolor[rgb]{0.56,0.35,0.01}{\textbf{\textit{#1}}}}
\newcommand{\ErrorTok}[1]{\textcolor[rgb]{0.64,0.00,0.00}{\textbf{#1}}}
\newcommand{\ExtensionTok}[1]{#1}
\newcommand{\FloatTok}[1]{\textcolor[rgb]{0.00,0.00,0.81}{#1}}
\newcommand{\FunctionTok}[1]{\textcolor[rgb]{0.00,0.00,0.00}{#1}}
\newcommand{\ImportTok}[1]{#1}
\newcommand{\InformationTok}[1]{\textcolor[rgb]{0.56,0.35,0.01}{\textbf{\textit{#1}}}}
\newcommand{\KeywordTok}[1]{\textcolor[rgb]{0.13,0.29,0.53}{\textbf{#1}}}
\newcommand{\NormalTok}[1]{#1}
\newcommand{\OperatorTok}[1]{\textcolor[rgb]{0.81,0.36,0.00}{\textbf{#1}}}
\newcommand{\OtherTok}[1]{\textcolor[rgb]{0.56,0.35,0.01}{#1}}
\newcommand{\PreprocessorTok}[1]{\textcolor[rgb]{0.56,0.35,0.01}{\textit{#1}}}
\newcommand{\RegionMarkerTok}[1]{#1}
\newcommand{\SpecialCharTok}[1]{\textcolor[rgb]{0.00,0.00,0.00}{#1}}
\newcommand{\SpecialStringTok}[1]{\textcolor[rgb]{0.31,0.60,0.02}{#1}}
\newcommand{\StringTok}[1]{\textcolor[rgb]{0.31,0.60,0.02}{#1}}
\newcommand{\VariableTok}[1]{\textcolor[rgb]{0.00,0.00,0.00}{#1}}
\newcommand{\VerbatimStringTok}[1]{\textcolor[rgb]{0.31,0.60,0.02}{#1}}
\newcommand{\WarningTok}[1]{\textcolor[rgb]{0.56,0.35,0.01}{\textbf{\textit{#1}}}}
\usepackage{graphicx}
\makeatletter
\def\maxwidth{\ifdim\Gin@nat@width>\linewidth\linewidth\else\Gin@nat@width\fi}
\def\maxheight{\ifdim\Gin@nat@height>\textheight\textheight\else\Gin@nat@height\fi}
\makeatother
% Scale images if necessary, so that they will not overflow the page
% margins by default, and it is still possible to overwrite the defaults
% using explicit options in \includegraphics[width, height, ...]{}
\setkeys{Gin}{width=\maxwidth,height=\maxheight,keepaspectratio}
% Set default figure placement to htbp
\makeatletter
\def\fps@figure{htbp}
\makeatother
\setlength{\emergencystretch}{3em} % prevent overfull lines
\providecommand{\tightlist}{%
  \setlength{\itemsep}{0pt}\setlength{\parskip}{0pt}}
\setcounter{secnumdepth}{-\maxdimen} % remove section numbering
\ifLuaTeX
  \usepackage{selnolig}  % disable illegal ligatures
\fi
\IfFileExists{bookmark.sty}{\usepackage{bookmark}}{\usepackage{hyperref}}
\IfFileExists{xurl.sty}{\usepackage{xurl}}{} % add URL line breaks if available
\urlstyle{same} % disable monospaced font for URLs
\hypersetup{
  pdftitle={Hausaufgabe 2},
  pdfauthor={Team 4 (Milena Mensching, Justus Weyers)},
  hidelinks,
  pdfcreator={LaTeX via pandoc}}

\title{Hausaufgabe 2}
\author{Team 4 (Milena Mensching, Justus Weyers)}
\date{2022-11-16}

\begin{document}
\maketitle

\hypertarget{aufgabe-1}{%
\section{Aufgabe 1}\label{aufgabe-1}}

\textbf{Sie wiegen mit einer digitalen Waage, deren kleinste
Schrittweite (d.h. Auflösung) 0,1 g beträgt, einen Apfel. Der Apfel
wiegt laut Anzeige 120,0 g. Stellen Sie sich vor, nach der Messung
stellen Sie fest, dass der Tisch auf dem die Waage steht, schief steht.
Die Schieflage beträgt ca. 1-2 Grad.}

\begin{itemize}
\tightlist
\item
  \textbf{Berechnen Sie welche systematische Abweichung dadurch
  entsteht.}
\item
  \textbf{Ist es richtig anzunehmen, dass die Messung dennoch korrekt
  war und die Messung aufgrund der Messunsicherheiten als richtig
  angenommen werden kann?}
\end{itemize}

\textbf{geg.:} Bestwert Masse \(m = 120,0\); Geräteart: digital;
Auflösung \(a=0,1g\). Als Winkel \(\alpha\) wird der Extremfall gewählt,
bei dem die Neigung 2° beträgt.

\textbf{ges.:} Systematische Abweichung \(e\)

\textbf{Rechnung:}

Bei einer schiefen Ebene nimmt der Betrag der Normalkraft um den Kosinus
des Neigunswinkels ab:
\[e = m*(1-cos(\alpha)) = 120,0g*(1-cos(2)) = 0,073g\]

Die Messunsicherheit beträgt:
\[ u = \frac{a}{2\sqrt{3}} \Rightarrow \frac{0,1g}{2\sqrt{3}} = 0,029g\]
Damit ist die Systematische Abweichung größer, als die Messunsicherheit.
Das Ergebnis ist also im Rahmen der Messunsicherheit nicht korrekt.

\hypertarget{aufgabe-2}{%
\section{Aufgabe 2:}\label{aufgabe-2}}

\textbf{Es wurden 3 verschiedene Techniken genutzt, um den Durchmesser
und Umfang eines runden Objektes zu messen und daraus experimentell die
Größe \(\pi\) zu bestimmen. Methode 1 ergibt: \(3,133 \pm 0,007\),
Methode 2 \(3,1609 \pm 0,0002\) und Methode 3 war \(3,14 \pm 0,03\).
Welche der 3 Methoden ist die präziseste? Begründen Sie Ihre Antwort.}

Methode 2 liefert das präziseste Ergebnis. Dieses wurde hier auf vier
Nachkommastellen genau angegeben. Die hohe Präzision sagt allerdings
nichts über die Richtigkeit des Ergebnisses aus.

\hypertarget{aufgabe-3}{%
\section{Aufgabe 3}\label{aufgabe-3}}

\textbf{Lara möchte das Volumen V einer Kugel bestimmen. Sie misst dafür
den Radius \(R\) der Kugel und bekommt \(R = (8,2 \pm 0,2)mm\). Sie
berechnet das Volumen und die Messunsicherheit des Volumens. Welches
Ergebnis bekommt sie?}

\(V_{Kugel}= \frac {4}{3}*\pi*R^3\)\\

\(Bestwert: V_{Kugel}= \frac {4}{3}*\pi*(8.2mm)^3=2309.565mm^3\)

\begin{Shaded}
\begin{Highlighting}[]
\NormalTok{(}\DecValTok{4}\SpecialCharTok{/}\DecValTok{3}\NormalTok{)}\SpecialCharTok{*}\NormalTok{pi}\SpecialCharTok{*}\FloatTok{8.2}\SpecialCharTok{**}\DecValTok{3}
\end{Highlighting}
\end{Shaded}

\begin{verbatim}
## [1] 2309.565
\end{verbatim}

\(Messunsicherheit:u=\vert\frac{\partial{V}}{\partial{R}}*u_R\vert=\vert4*\pi*R^2*u_R\vert= 4*\pi*(8.2mm)^2*0.2mm \approx 170mm^3\)

\begin{Shaded}
\begin{Highlighting}[]
\DecValTok{4}\SpecialCharTok{*}\NormalTok{pi}\SpecialCharTok{*}\FloatTok{8.2}\SpecialCharTok{**}\DecValTok{2}\SpecialCharTok{*}\FloatTok{0.2}
\end{Highlighting}
\end{Shaded}

\begin{verbatim}
## [1] 168.9926
\end{verbatim}

\(\rightarrow V_{Kugel}= (2310\pm 170)mm^3\)

\hypertarget{aufgabe-4}{%
\section{Aufgabe 4}\label{aufgabe-4}}

\textbf{Tom hängt drei Gewichte an eine Feder. Er benutzt eine Waage
(die Messunsicherheit des Gerätes \(u_{Waage}\) beträgt 0,05g, die
Auflösung der Waage beträgt 0,1g) um die drei Massen einmalig zu
bestimmen. Tom liest auf der Waage folgenden Werte für die Massen ab:}

\begin{itemize}
\tightlist
\item
  \(M_1= 30,2 g\)
\item
  \(M_2= 9,8 g\)
\item
  \(M_3= 5,1 g\)
\end{itemize}

\textbf{Wie schwer ist die gesamte Masse? Vergessen Sie nicht die
Messunsicherheit dazu.}

Berechnung der Skalen-Unsicherheit \(u_{Skala}\) mit \(a=0,1g\):
\[u_{Skala} = \frac{a}{2\sqrt{6}} = \frac{0,1g}{2\sqrt{6}}\] Berechnung
der Geräteunsicherheit \(u_{Gerät}\):
\[u_{Gerät} =\pm \sqrt{u_{Skala}^2+u_{Waage}^2} =\pm \sqrt{(\frac{0,1}{2\sqrt{6}})^2+(0,05)^2}g = \pm 0,054g\]

Formel für die Berechnung der Gesamtmasse \(m_{ges}\):
\[m_{ges} = m_1+m_2+m_3\]

Berechnung der Messunsicherheit \(u\): \begin{align*}
\begin{split}
u&=\pm \sqrt{(\frac{\partial m_{ges}}{\partial m_1}*u_{Gerät})^2+(\frac{\partial m_{ges}}{\partial m_2}*u_{Gerät})^2+(\frac{\partial m_{ges}}{\partial m_3}*u_{Gerät})^2}\\
 &= \sqrt{((1+m_2+m_3)*u_{Gerät})^2+((m_1+1+m_3)*u_{Gerät})^2+((m_1+m_2+1)*u_{Gerät})^2}\\
\Rightarrow &= \sqrt{((1+9,8+5,1)*0,054)^2+((30,2+1+5,1)*0,054)^2+((30,2+9,8+1)*0,054)^2}\\
&= 3,1g\\
\end{split}
\end{align*}

Die Gesamtmasse \(m_{ges}\) beträgt also:
\[m_{ges}=((30,2+9,8+5,1)\pm 3,1)g = (45,1 \pm 3,1)g\]

\hypertarget{aufgabe-5}{%
\section{Aufgabe 5}\label{aufgabe-5}}

\textbf{Lisa und Yamile messen die Höhe eines Gebäudes mit
unterschiedlichen Methoden. Sie bekommen folgende Ergebnisse:}

\begin{itemize}
\tightlist
\item
  \(h_{Lisa}= (12,25 \pm 0,25)m\)
\item
  \(h_{Yamile}= (14,5 \pm 0,5)m\)
\end{itemize}

\textbf{Ist der Unterschied zwischen Messung 1 und 2 signifikant?
Begründen Sie Ihre Antwort.}

\(\Rightarrow\) Der Unterschied zwischen den beiden Messungen ist
signifikant, da sich die beiden Messintervalle nicht überlappen.

\textbf{Gibt es vermutlich eine systematische Abweichung in einer der
beiden Methoden?}

\(\Rightarrow\) Es gibt wahrscheinlich bei mindestens einer der
Messungen eine systematische Abweichung, da sich beide Intervalle trotz
des gleichen zu messenden Gebäudes nicht überlappen

\hypertarget{aufgabe-6}{%
\section{Aufgabe 6}\label{aufgabe-6}}

\textbf{Jo will die Periodendauer T eines Pendels experimentell
bestimmen. Jo entscheidet statt einer einzelnen Periode 20 Perioden mit
der Stoppuhr zu messen. Wie kann Jo die Periodendauer T und ihre
Messunsicherheit bestimmen? Wenn Sie jetzt die Ergebnisse dieser Übung
reflektieren, würden Sie sagen dass es um die Periodendauern eines
Pendels zu bestimmen, besser ist mehrere Perioden zu messen oder
wenige?}

Es handelt sich bei diesem Vorgehen um eine Auswertung der Daten nach
Typ-A (GUM), also einen statistschen Ansatz zur Abschätzung der
Messunsicherheit.

Die Verteilung der Messergebnisse wird als normalverteilt angenommen
(Gauss-Verteilung).

Jo kann den Mittelwert \(\overline{T}\) aus allen 20 Periodendauern
berechnen. \[\overline{T} = \frac{1}{n}\sum \limits_{i=1}^nT_i\]

\begin{itemize}
\tightlist
\item
  \(\bar{T}\): Gemittelte Periodendauer
\item
  \(T_i\): Periodendauer der \(i\)-ten Messung
\item
  \(n\): Anzahl von Perioden
\end{itemize}

Um die Streuung der Messwerte einschätzen zu können wird dann die
Standardabweichung berechnet:
\[\sigma_x = \sqrt{\frac{1}{n-1} \sum_{i=1}^n (T_i - \overline{T})^2}\]
Diese sagt aus, wie stark die Ergebnisse um den Mittelwert streuen. Da
der Mittelwert für eine Teilmenge der Messungen ebenfalls streuen kann
wird auch die Standardabweichung des Mittelwertes berechnet:
\[\sigma_{\overline{T}}=\frac{\sigma_x}{\sqrt{n}}\] Diese ist für
normalverteilte Stichproben der entscheidende Parameter. ``Dieser gibt
das Intervall an, in dem sich 68,3\% aller Mittelwerte, noch folgender
Messreihen befinden werden'' {[}Skript: ``Umgang mit Messunsicherheiten
Teil 1'', S.5{]}. Die \textbf{Standardmessunsicherheit} der n-mal
geessenen Periodendauer lässt sich so statistisch abschätzen {[}ebd.{]}.

Das Messergebnis wird mit \(\overline{T} \pm \sigma_{\overline{T}}\)
angegeben.

Aus der Formel für die Standardabweichung des Mittelwertes wird
ersichtlich, dass für eine größere Anzahl von Messungen n die
Standardabweichung des Mittelwertes kleiner wird.

\hypertarget{aufgabe-7}{%
\section{Aufgabe 7:}\label{aufgabe-7}}

\textbf{Wieso ist die Messung der Periodendauer eines Pendels präziser,
wenn man die Stoppuhr beim Punkt der maximalen Geschwindigkeit des
Pendels und nicht beim Punkt der minimalen Geschwindigkeit startet (bzw.
stoppt)? Hinweis: denken Sie an die Reaktionszeit einer Person.}

Vermutlich ist das der Unsicherheit eines Menschan bei der Festlegung
auf einen Zeitpunkt der maximalen Auslenkung des Pendels geschuldet. Die
Unsicherheit besteht dann darin, dass es schwierig ist, in dem
vergleichsweise langen Zeitraum, den das Pendel in der maximalen
Auslenkung und anderen Positionen der Fast-Maximalauslenkung verbringt,
die Stoppuhr richtig zu betätigen. Die Periodendauer kann dann auf
zufällige Art und Weise als zu kurz oder zu lang angegeben werden.
Dieser Umstand muss nicht zu einer größeren systematischen Abweichung
führen, allerdings zu einer kleineren Präzision. Der Durchgang durch die
minimale Auslenkung ist dagegen vergleichsweise gut auszumachen und
anzugeben. Die Reaktionszeit ist nachträglich aber in beiden Fällen zu
berücksichtigen.

Es wären aber auch noch präzisere Methoden zur Bestimmung der
Periodendauer denkbar. Beispielsweise die Aufnahme des schwingenden
Pendels mit einer Kamera mit einer möglichst hohen zeitlichen Auflösung.
Am Computer ließen sich dann der Zeitpunkt des Einzelbildes der
minimalen Auslenkung bestimmen. Die Reaktionszeit würde in diesem Falle
praktisch wegfallen.

\hypertarget{aufgabe-8}{%
\section{Aufgabe 8}\label{aufgabe-8}}

\textbf{Lina, die \((60,0 \pm 0,5)kg\) wiegt, fragt sich, welche
kinetische Energie sie hat, wenn sie sprintet. Sie hat kein
Geschwindigkeitsmessgerät, überlegt sich aber, dass sie aus einer festen
Strecke und aus der Zeit, die sie für diese braucht, ihre
Geschwindigkeit berechnen kann. Lina bittet einen Freund, ihre Zeit bei
einem kurzen 10m-Sprint zu messen und erhält folgende Ergebnisse:}

\begin{itemize}
\tightlist
\item
  1,56s
\item
  1,34s
\item
  1,44s
\item
  1,50s
\item
  1,38s
\end{itemize}

\textbf{Die 10m konnte Lina nur abschätzen, sodass die Strecke
tatsächlich \((10 \pm 1)m\) lang ist.}

\textbf{Bestimmen Sie zunächst die durchschnittliche Zeit, die Lina für
10m braucht, und ihre Unsicherheit. Der t-Wert für 5 Messwerte ist 1,11.
Berechnen Sie dann die kinetische Energie, die Lina im Mittel beim
Sprinten hat. Berechnen Sie auch die Unsicherheit dieses Ergebnisses.}

Durchschnittliche Zeit und ihre Unsicherheit:

Mittelwert: \(\bar{x} = \frac {1}{n}\sum_{i}^n x_i\)

Standardabweichung:
\(\sigma = \sqrt{\frac{1}{n-1}\sum_{i}^n (x_i - \bar{x})^2}\)

Standardabweichung des Mittelwerts:
\(\sigma_{\bar{x}}= \frac{\sigma}{\sqrt{n}}\)

Vertrauensabweichung bei kleinen Stichprobenumfämgen:
\(\epsilon_{\bar{x}}= \frac{t*\sigma_x}{\sqrt(n)}\) ; mit dem
Studentfaktor \(t(5)=1,11\)

\begin{Shaded}
\begin{Highlighting}[]
\CommentTok{\# Eingabe der Zeiten}
\NormalTok{time }\OtherTok{\textless{}{-}} \FunctionTok{c}\NormalTok{(}\FloatTok{1.56}\NormalTok{, }\FloatTok{1.34}\NormalTok{, }\FloatTok{1.44}\NormalTok{, }\FloatTok{1.50}\NormalTok{, }\FloatTok{1.38}\NormalTok{)}

\CommentTok{\#Anzahl von Messungen}
\NormalTok{n }\OtherTok{\textless{}{-}} \DecValTok{5}

\CommentTok{\# Ausgabe}
\FunctionTok{data.frame}\NormalTok{(Kenngröße}\OtherTok{=}\FunctionTok{c}\NormalTok{(}\StringTok{"Mittelwert"}\NormalTok{, }
                       \StringTok{"Standardabweichung"}\NormalTok{, }
                       \StringTok{"Vertrauensabweichung"}\NormalTok{), }
           \AttributeTok{Wert=}\FunctionTok{c}\NormalTok{(}\FunctionTok{mean}\NormalTok{(time), }\FunctionTok{sd}\NormalTok{(time), }\FloatTok{1.11}\SpecialCharTok{*}\FunctionTok{sd}\NormalTok{(time)}\SpecialCharTok{/}\FunctionTok{sqrt}\NormalTok{(n)))}
\end{Highlighting}
\end{Shaded}

\begin{verbatim}
##              Kenngröße       Wert
## 1           Mittelwert 1.44400000
## 2   Standardabweichung 0.08876936
## 3 Vertrauensabweichung 0.04406574
\end{verbatim}

Durchschnittliche Zeit und Unsicherheit: \(t=(1,444\pm 0,044)s\).

Damit sind alle Größen zur Berechnung der kinetischen Energie (m, s und
t) samt deren Unsicherheiten zusammengetragen. Die Kinetische Energie
und deren Unsicherheit berechnet sich zu: \begin{align*}
E_{kin} &= \frac{1}{2}\cdot m\cdot (\frac{s}{t})^2\\
& = \frac{1}{2}\cdot 60kg\cdot (\frac{10m}{1,444s})^2\\
&=1440.747J
\end{align*}

\begin{align*}
u_{E} &= \sqrt{(\frac{\partial E}{\partial s}*u_{s})^2+(\frac{\partial E}{\partial m}*u_{m})^2+(\frac{\partial E}{\partial t}*u_{t})^2}\\
 &= \sqrt{(\frac{1.0*m*s}{t^{2}}*u_{s})^2+(\frac{0.5*s^{2}}{t^{2}}*u_{m})^2+(-\frac{1.0*m*s^{2}}{t^{3}}*u_{t})^2}\\
 &= \sqrt{(\frac{1.0*60.0*10.0}{1.444^{2}}*1)^2+(\frac{0.5*10^{2}}{1,444^{2}}*0,5)^2+(-\frac{1.0*60*10^{2}}{1,444^{3}}*0,044)^2}J\\
 &=300J
\end{align*}

Linas kinetische Energie beträgt also:
\(E_{kin}=(1440 \pm 300)J = (1,44\pm0,30)kJ\).

\end{document}

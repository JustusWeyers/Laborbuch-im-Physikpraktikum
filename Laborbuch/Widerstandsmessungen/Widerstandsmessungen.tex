% Options for packages loaded elsewhere
\PassOptionsToPackage{unicode}{hyperref}
\PassOptionsToPackage{hyphens}{url}
%
\documentclass[
  9pt,
]{article}
\usepackage{amsmath,amssymb}
\usepackage{lmodern}
\usepackage{iftex}
\ifPDFTeX
  \usepackage[T1]{fontenc}
  \usepackage[utf8]{inputenc}
  \usepackage{textcomp} % provide euro and other symbols
\else % if luatex or xetex
  \usepackage{unicode-math}
  \defaultfontfeatures{Scale=MatchLowercase}
  \defaultfontfeatures[\rmfamily]{Ligatures=TeX,Scale=1}
\fi
% Use upquote if available, for straight quotes in verbatim environments
\IfFileExists{upquote.sty}{\usepackage{upquote}}{}
\IfFileExists{microtype.sty}{% use microtype if available
  \usepackage[]{microtype}
  \UseMicrotypeSet[protrusion]{basicmath} % disable protrusion for tt fonts
}{}
\makeatletter
\@ifundefined{KOMAClassName}{% if non-KOMA class
  \IfFileExists{parskip.sty}{%
    \usepackage{parskip}
  }{% else
    \setlength{\parindent}{0pt}
    \setlength{\parskip}{6pt plus 2pt minus 1pt}}
}{% if KOMA class
  \KOMAoptions{parskip=half}}
\makeatother
\usepackage{xcolor}
\IfFileExists{xurl.sty}{\usepackage{xurl}}{} % add URL line breaks if available
\IfFileExists{bookmark.sty}{\usepackage{bookmark}}{\usepackage{hyperref}}
\hypersetup{
  pdftitle={Widerstandsmessungen},
  pdfauthor={Milena Mensching, Justus Weyers},
  pdflang={de},
  hidelinks,
  pdfcreator={LaTeX via pandoc}}
\urlstyle{same} % disable monospaced font for URLs
\usepackage[margin=1in]{geometry}
\usepackage{color}
\usepackage{fancyvrb}
\newcommand{\VerbBar}{|}
\newcommand{\VERB}{\Verb[commandchars=\\\{\}]}
\DefineVerbatimEnvironment{Highlighting}{Verbatim}{commandchars=\\\{\}}
% Add ',fontsize=\small' for more characters per line
\usepackage{framed}
\definecolor{shadecolor}{RGB}{248,248,248}
\newenvironment{Shaded}{\begin{snugshade}}{\end{snugshade}}
\newcommand{\AlertTok}[1]{\textcolor[rgb]{0.94,0.16,0.16}{#1}}
\newcommand{\AnnotationTok}[1]{\textcolor[rgb]{0.56,0.35,0.01}{\textbf{\textit{#1}}}}
\newcommand{\AttributeTok}[1]{\textcolor[rgb]{0.77,0.63,0.00}{#1}}
\newcommand{\BaseNTok}[1]{\textcolor[rgb]{0.00,0.00,0.81}{#1}}
\newcommand{\BuiltInTok}[1]{#1}
\newcommand{\CharTok}[1]{\textcolor[rgb]{0.31,0.60,0.02}{#1}}
\newcommand{\CommentTok}[1]{\textcolor[rgb]{0.56,0.35,0.01}{\textit{#1}}}
\newcommand{\CommentVarTok}[1]{\textcolor[rgb]{0.56,0.35,0.01}{\textbf{\textit{#1}}}}
\newcommand{\ConstantTok}[1]{\textcolor[rgb]{0.00,0.00,0.00}{#1}}
\newcommand{\ControlFlowTok}[1]{\textcolor[rgb]{0.13,0.29,0.53}{\textbf{#1}}}
\newcommand{\DataTypeTok}[1]{\textcolor[rgb]{0.13,0.29,0.53}{#1}}
\newcommand{\DecValTok}[1]{\textcolor[rgb]{0.00,0.00,0.81}{#1}}
\newcommand{\DocumentationTok}[1]{\textcolor[rgb]{0.56,0.35,0.01}{\textbf{\textit{#1}}}}
\newcommand{\ErrorTok}[1]{\textcolor[rgb]{0.64,0.00,0.00}{\textbf{#1}}}
\newcommand{\ExtensionTok}[1]{#1}
\newcommand{\FloatTok}[1]{\textcolor[rgb]{0.00,0.00,0.81}{#1}}
\newcommand{\FunctionTok}[1]{\textcolor[rgb]{0.00,0.00,0.00}{#1}}
\newcommand{\ImportTok}[1]{#1}
\newcommand{\InformationTok}[1]{\textcolor[rgb]{0.56,0.35,0.01}{\textbf{\textit{#1}}}}
\newcommand{\KeywordTok}[1]{\textcolor[rgb]{0.13,0.29,0.53}{\textbf{#1}}}
\newcommand{\NormalTok}[1]{#1}
\newcommand{\OperatorTok}[1]{\textcolor[rgb]{0.81,0.36,0.00}{\textbf{#1}}}
\newcommand{\OtherTok}[1]{\textcolor[rgb]{0.56,0.35,0.01}{#1}}
\newcommand{\PreprocessorTok}[1]{\textcolor[rgb]{0.56,0.35,0.01}{\textit{#1}}}
\newcommand{\RegionMarkerTok}[1]{#1}
\newcommand{\SpecialCharTok}[1]{\textcolor[rgb]{0.00,0.00,0.00}{#1}}
\newcommand{\SpecialStringTok}[1]{\textcolor[rgb]{0.31,0.60,0.02}{#1}}
\newcommand{\StringTok}[1]{\textcolor[rgb]{0.31,0.60,0.02}{#1}}
\newcommand{\VariableTok}[1]{\textcolor[rgb]{0.00,0.00,0.00}{#1}}
\newcommand{\VerbatimStringTok}[1]{\textcolor[rgb]{0.31,0.60,0.02}{#1}}
\newcommand{\WarningTok}[1]{\textcolor[rgb]{0.56,0.35,0.01}{\textbf{\textit{#1}}}}
\usepackage{graphicx}
\makeatletter
\def\maxwidth{\ifdim\Gin@nat@width>\linewidth\linewidth\else\Gin@nat@width\fi}
\def\maxheight{\ifdim\Gin@nat@height>\textheight\textheight\else\Gin@nat@height\fi}
\makeatother
% Scale images if necessary, so that they will not overflow the page
% margins by default, and it is still possible to overwrite the defaults
% using explicit options in \includegraphics[width, height, ...]{}
\setkeys{Gin}{width=\maxwidth,height=\maxheight,keepaspectratio}
% Set default figure placement to htbp
\makeatletter
\def\fps@figure{htbp}
\makeatother
\setlength{\emergencystretch}{3em} % prevent overfull lines
\providecommand{\tightlist}{%
  \setlength{\itemsep}{0pt}\setlength{\parskip}{0pt}}
\setcounter{secnumdepth}{-\maxdimen} % remove section numbering
\ifLuaTeX
\usepackage[bidi=basic]{babel}
\else
\usepackage[bidi=default]{babel}
\fi
\babelprovide[main,import]{ngerman}
% get rid of language-specific shorthands (see #6817):
\let\LanguageShortHands\languageshorthands
\def\languageshorthands#1{}
\ifLuaTeX
  \usepackage{selnolig}  % disable illegal ligatures
\fi

\title{Widerstandsmessungen}
\author{Milena Mensching, Justus Weyers}
\date{2023-01-05}

\begin{document}
\maketitle

\hypertarget{experiment}{%
\section{Experiment}\label{experiment}}

\hypertarget{thema}{%
\subsection{Thema}\label{thema}}

Bestimmung von Widerständen auf direkte und indirekte Weise

\hypertarget{material}{%
\subsection{Material}\label{material}}

\begin{itemize}
\item{2 Multimeter}
\item{Breadboard}
\item{Kabel}
\item{Netzgerät}
\end{itemize}

\hypertarget{auslesung-der-widerstuxe4nde}{%
\subsection{Auslesung der
Widerstände}\label{auslesung-der-widerstuxe4nde}}

Die Farbreihenfolge auf dem ersten Widerstand ist:

\(Braun, Schwarz, Schwarz, Gelb, Braun\)

Dementsprechend kodiert der Widerstand für

\begin{equation*}
R = (100 \cdot 10^4\pm 1\% ) \Omega \Leftrightarrow R=(1\pm 0.01) M\Omega
\end{equation*}

Der zweite Widerstand ist als Widerstand von \(1\Omega\) gekennzeichnet

\hypertarget{direkte-messung}{%
\subsection{Direkte Messung}\label{direkte-messung}}

Messen der zwei Widerstände liefert Werte von:

\begin{equation*}
\begin{split}
1.Widerstand: R &= 1,005 M\Omega \\
2. Widerstand: R &= 1,0 \Omega
\end{split}
\end{equation*}

\hypertarget{messunsicherheiten}{%
\subsection{Messunsicherheiten}\label{messunsicherheiten}}

Die Messunsicherheit der direkten Messung ergibt sich aus der
Messunsicherheit der digitalen Skala

\begin{equation*}
\begin{split}
u_{Skala} &=\frac{a}{2*\sqrt{3}} \\
1.Widerstand: u_1 &= \frac{0,001M\Omega}{2*\sqrt{3}} \approx \pm 0,00029 M\Omega \\
2.Widerstand: u_2 &= \frac{0,1\Omega}{2*\sqrt{3}} \approx \pm 0,029\Omega 
\end{split}
\end{equation*}

\begin{Shaded}
\begin{Highlighting}[]
\FloatTok{0.001}\SpecialCharTok{/}\NormalTok{(}\DecValTok{2}\SpecialCharTok{*}\FunctionTok{sqrt}\NormalTok{(}\DecValTok{3}\NormalTok{)) }\CommentTok{\#Unsicherheit 1}
\end{Highlighting}
\end{Shaded}

\begin{verbatim}
## [1] 0.0002886751
\end{verbatim}

\begin{Shaded}
\begin{Highlighting}[]
\FloatTok{0.1}\SpecialCharTok{/}\NormalTok{(}\DecValTok{2}\SpecialCharTok{*}\FunctionTok{sqrt}\NormalTok{(}\DecValTok{3}\NormalTok{)) }\CommentTok{\#Unsicherheit 2}
\end{Highlighting}
\end{Shaded}

\begin{verbatim}
## [1] 0.02886751
\end{verbatim}

\hypertarget{indirekte-messungen}{%
\section{Indirekte Messungen}\label{indirekte-messungen}}

\hypertarget{aufbau-a}{%
\subsection{Aufbau (a)}\label{aufbau-a}}

Mithilfe der Kabel und des Breadboards wurde folgender Stromkreis
aufgebaut:

Folgende Werte wurden gemessen:

\begin{itemize}
\item {Widerstand 1 (Messung bei 5,0V)}
\begin{itemize}
\item {Spannung: 5,04V}
\item {5,4 $\mu$A}
\end{itemize}
\item {Widerstand 2 (Messung bei 0,6V)}
\begin{itemize}
\item {Spannung: 0,15 V}
\item {150,3 mA}
\end{itemize}
\end{itemize}

\hypertarget{messunsicherheiten-1}{%
\subsubsection{Messunsicherheiten}\label{messunsicherheiten-1}}

Die Messunsicherheitem von Spannung und Strom ergeben sich aus der
Messunsicherheit der digitalen Skala

\begin{equation*}
\begin{split}
u_{Skala} &=\frac{a}{2*\sqrt{3}} \\
\underline{1.Widerstand:} \\
Spannung: u_{U1a} &= \frac{0,01V}{2*\sqrt{3}} \approx \pm 0,0029 V \\
Strom: u_{I1a} &= \frac{0,1\mu A}{2*\sqrt{3}} \approx \pm 0,029\mu A \\
\underline{2.Widerstand:} \\
Spannung: u_{U2a} &= \frac{0,01V}{2*\sqrt{3}} \approx \pm 0,0029 V \\
Strom: u_{I2a} &= \frac{0,1mA}{2*\sqrt{3}} \approx \pm 0,029 mA \\
\end{split}
\end{equation*}

\begin{Shaded}
\begin{Highlighting}[]
\FloatTok{0.01}\SpecialCharTok{/}\NormalTok{(}\DecValTok{2}\SpecialCharTok{*}\FunctionTok{sqrt}\NormalTok{(}\DecValTok{3}\NormalTok{)) }\CommentTok{\#Unsicherheit Spannung 1,2}
\end{Highlighting}
\end{Shaded}

\begin{verbatim}
## [1] 0.002886751
\end{verbatim}

\begin{Shaded}
\begin{Highlighting}[]
\FloatTok{0.1}\SpecialCharTok{/}\NormalTok{(}\DecValTok{2}\SpecialCharTok{*}\FunctionTok{sqrt}\NormalTok{(}\DecValTok{3}\NormalTok{)) }\CommentTok{\#Unsicherheit Strom 1,2}
\end{Highlighting}
\end{Shaded}

\begin{verbatim}
## [1] 0.02886751
\end{verbatim}

Folglich liegen die gemessenen Größen bei:

\begin{itemize}
\item $U1a = (5,0400 \pm 0,0029)V$
\item $I1a = (5,400 \pm 0,029) \mu A$
\item $U2a = (0,1500 \pm 0,0029)V$
\item $I2a = (150,300 \pm 0,029) mA$
\end{itemize}

\hypertarget{berechnung-der-widerstuxe4nde}{%
\subsection{Berechnung der
Widerstände}\label{berechnung-der-widerstuxe4nde}}

Es gilt das Ohmsche Gesetz:

\(U=I\cdot R \Leftrightarrow R = \frac{U}{I}\)

\noindent \(R\) =Widerstand\\
\noindent \(U\) = Spannung\\
\noindent \(I\) = Strom

Für die Widerstände ergeben sich somit für Aufbau (a) Bestwerte von:
\begin{equation*}
\begin{split}
1. Widerstand: R=\frac{U}{I} = \frac {5,0400V}{5,4 \mu A} \approx 933333,3 \Omega \\
2.Widerstand: R=\frac{U}{I} = \frac {0,1500V}{150,3 mA} \approx 0,998004 \Omega \\
\end{split}
\end{equation*}

\begin{Shaded}
\begin{Highlighting}[]
\NormalTok{(}\FloatTok{5.04}\SpecialCharTok{/}\NormalTok{(}\FloatTok{5.4}\SpecialCharTok{*}\DecValTok{10}\SpecialCharTok{**{-}}\DecValTok{6}\NormalTok{)) }\CommentTok{\#Widerstand 1}
\end{Highlighting}
\end{Shaded}

\begin{verbatim}
## [1] 933333.3
\end{verbatim}

\begin{Shaded}
\begin{Highlighting}[]
\NormalTok{(}\FloatTok{0.15}\SpecialCharTok{/}\NormalTok{(}\FloatTok{150.3}\SpecialCharTok{*}\DecValTok{10}\SpecialCharTok{**{-}}\DecValTok{3}\NormalTok{)) }\CommentTok{\#Widerstand 2}
\end{Highlighting}
\end{Shaded}

\begin{verbatim}
## [1] 0.998004
\end{verbatim}

\hypertarget{messunsicherheiten-2}{%
\subsubsection{Messunsicherheiten}\label{messunsicherheiten-2}}

Die Messunsicherheit für die indirekt bestimmten Widerstände ergibt sich
mit folgender Formel:

\begin {equation*}
\begin{split}
u_R &= \sqrt{\left (\frac{\partial R}{\partial U} \cdot u_U\right )^2 + \left (\frac{\partial R}{\partial I} \cdot u_I\right )^2 } \\
u_R &= \sqrt{\left (\frac{1}{I} \cdot u_U\right )^2 + \left (\frac{-U}{I^2} \cdot u_I\right )^2 } \\
1.Widerstand: u_{R1a}&= \sqrt{\left (\frac{1}{5,4\mu A} \cdot 0,0029V\right )^2 + \left (\frac{-5,04V}{5,4\mu A^2} \cdot 0,029 \mu A\right )^2 } \approx \pm 5000\Omega \\
2.Widerstand: u_{R2a}&= \sqrt{\left (\frac{1}{150,300mA} \cdot 0,0029V \right )^2 + \left (\frac{0,1500V}{150,300mA^2} \cdot 0,029mA\right )^2 } \approx \pm 0,019 \Omega
\end{split}
\end{equation*}

\begin{Shaded}
\begin{Highlighting}[]
\FunctionTok{sqrt}\NormalTok{(((}\DecValTok{1}\SpecialCharTok{/}\NormalTok{(}\FloatTok{5.4}\SpecialCharTok{*}\DecValTok{10}\SpecialCharTok{**{-}}\DecValTok{6}\NormalTok{))}\SpecialCharTok{*}\FloatTok{0.0029}\NormalTok{)}\SpecialCharTok{**}\DecValTok{2}\SpecialCharTok{+}\NormalTok{(}\FloatTok{5.04}\SpecialCharTok{/}\NormalTok{((}\FloatTok{5.4}\SpecialCharTok{*}\DecValTok{10}\SpecialCharTok{**{-}}\DecValTok{6}\NormalTok{)}\SpecialCharTok{**}\DecValTok{2}\NormalTok{)}\SpecialCharTok{*}\FloatTok{0.029}\SpecialCharTok{*}\DecValTok{10}\SpecialCharTok{**{-}}\DecValTok{6}\NormalTok{)}\SpecialCharTok{**}\DecValTok{2}\NormalTok{) }\CommentTok{\#Unsicherheit Widerstand 1}
\end{Highlighting}
\end{Shaded}

\begin{verbatim}
## [1] 5041.033
\end{verbatim}

\begin{Shaded}
\begin{Highlighting}[]
\FunctionTok{sqrt}\NormalTok{(((}\DecValTok{1}\SpecialCharTok{/}\NormalTok{(}\FloatTok{150.3}\SpecialCharTok{*}\DecValTok{10}\SpecialCharTok{**{-}}\DecValTok{3}\NormalTok{))}\SpecialCharTok{*}\FloatTok{0.0029}\NormalTok{)}\SpecialCharTok{**}\DecValTok{2}\SpecialCharTok{+}\NormalTok{(}\FloatTok{0.15}\SpecialCharTok{/}\NormalTok{((}\FloatTok{150.3}\SpecialCharTok{*}\DecValTok{10}\SpecialCharTok{**{-}}\DecValTok{3}\NormalTok{)}\SpecialCharTok{**}\DecValTok{2}\NormalTok{)}\SpecialCharTok{*}\FloatTok{0.029}\SpecialCharTok{*}\DecValTok{10}\SpecialCharTok{**{-}}\DecValTok{3}\NormalTok{)}\SpecialCharTok{**}\DecValTok{2}\NormalTok{) }\CommentTok{\#Unsicherheit Widerstand 2}
\end{Highlighting}
\end{Shaded}

\begin{verbatim}
## [1] 0.0192957
\end{verbatim}

Die Widerstände für den Aufbau (a) ergeben sich somit insgesamt zu:

\begin{itemize}
\item $R1a = (93000 \pm 5000)\Omega $
\item $R2a = (0,998 \pm 0,019) \Omega$
\end{itemize}

\hypertarget{aufbau-b}{%
\subsection{Aufbau (b)}\label{aufbau-b}}

Anschließend wurde der Schaltkreis entsprechend umgebaut:

Folgende Werte wurden gemessen:

\begin{itemize}
\item {Widerstand 1 (Messung bei 5,0V)}
\begin{itemize}
\item {Spannung: 5,03V}
\item {Strom: 5,0 $\mu$A}
\end{itemize}
\item {Widerstand 2 (Messung bei 0,8V)}
\begin{itemize}
\item {Spannung: 0,4 V}
\item {Strom: 101,8 mA}
\end{itemize}
\end{itemize}

\hypertarget{messunsicherheiten-3}{%
\subsubsection{Messunsicherheiten}\label{messunsicherheiten-3}}

Die Messunsicherheiten von Spannung und Strom sind identisch zu denen im
Aufbau (a). Folglich liegen die gemessenen Größen bei:

\begin{itemize}
\item $U1b = (5,0300 \pm 0,0029)V$
\item $I1b = (5,000 \pm 0,029) \mu A$
\item $U2b = (0,4000 \pm 0,0029)V$
\item $I2b = (101,800 \pm 0,029) mA$
\end{itemize}

\hypertarget{berechnung-der-widerstuxe4nde-1}{%
\subsection{Berechnung der
Widerstände}\label{berechnung-der-widerstuxe4nde-1}}

Es gilt das Ohmsche Gesetz:

\(U=I\cdot R \Leftrightarrow R = \frac{U}{I}\)

\noindent \(R\) =Widerstand\\
\noindent \(U\) = Spannung\\
\noindent \(I\) = Strom

Für die Widerstände ergeben sich somit für Aufbau (b) Bestwerte von:
\begin{equation*}
\begin{split}
1. Widerstand: R&=\frac{U}{I} = \frac {5,0300V}{5,000 \mu A} \approx 1006000 \Omega \\
2.Widerstand: R&=\frac{U}{I} = \frac {0,4000V}{101,800 mA} \approx 3,929273 \Omega \\
\end{split}
\end{equation*}

\begin{Shaded}
\begin{Highlighting}[]
\NormalTok{(}\FloatTok{5.03}\SpecialCharTok{/}\NormalTok{(}\DecValTok{5}\SpecialCharTok{*}\DecValTok{10}\SpecialCharTok{**{-}}\DecValTok{6}\NormalTok{)) }\CommentTok{\#Widerstand 1}
\end{Highlighting}
\end{Shaded}

\begin{verbatim}
## [1] 1006000
\end{verbatim}

\begin{Shaded}
\begin{Highlighting}[]
\NormalTok{(}\FloatTok{0.4}\SpecialCharTok{/}\NormalTok{(}\FloatTok{101.8}\SpecialCharTok{*}\DecValTok{10}\SpecialCharTok{**{-}}\DecValTok{3}\NormalTok{)) }\CommentTok{\#Widerstand 2}
\end{Highlighting}
\end{Shaded}

\begin{verbatim}
## [1] 3.929273
\end{verbatim}

\hypertarget{messunsicherheiten-4}{%
\subsubsection{Messunsicherheiten}\label{messunsicherheiten-4}}

Die Messunsicherheit für die indirekt bestimmten Widerstände ergibt sich
mit folgender Formel:

\begin {equation*}
\begin{split}
u_R &= \sqrt{\left (\frac{\partial R}{\partial U} \cdot u_U\right )^2 + \left (\frac{\partial R}{\partial I} \cdot u_I\right )^2 } \\
u_R &= \sqrt{\left (\frac{1}{I} \cdot u_U\right )^2 + \left (\frac{-U}{I^2} \cdot u_I\right )^2 } \\
1.Widerstand: u_{R1b}&= \sqrt{\left (\frac{1}{5,000\mu A} \cdot 0,0029V\right )^2 + \left (\frac{-5,0300V}{5,000\mu A^2} \cdot 0,029 \mu A\right )^2 } \approx \pm 5900\Omega \\
2.Widerstand: u_{R2b}&= \sqrt{\left (\frac{1}{101,800mA} \cdot 0,0029V \right )^2 + \left (\frac{0,4000V}{101,800mA^2} \cdot 0,029mA\right )^2 } \approx \pm 0,029\Omega
\end{split}
\end{equation*}

\begin{Shaded}
\begin{Highlighting}[]
\FunctionTok{sqrt}\NormalTok{(((}\DecValTok{1}\SpecialCharTok{/}\NormalTok{(}\DecValTok{5}\SpecialCharTok{*}\DecValTok{10}\SpecialCharTok{**{-}}\DecValTok{6}\NormalTok{))}\SpecialCharTok{*}\FloatTok{0.0029}\NormalTok{)}\SpecialCharTok{**}\DecValTok{2}\SpecialCharTok{+}\NormalTok{(}\FloatTok{5.03}\SpecialCharTok{/}\NormalTok{((}\DecValTok{5}\SpecialCharTok{*}\DecValTok{10}\SpecialCharTok{**{-}}\DecValTok{6}\NormalTok{)}\SpecialCharTok{**}\DecValTok{2}\NormalTok{)}\SpecialCharTok{*}\FloatTok{0.029}\SpecialCharTok{*}\DecValTok{10}\SpecialCharTok{**{-}}\DecValTok{6}\NormalTok{)}\SpecialCharTok{**}\DecValTok{2}\NormalTok{) }\CommentTok{\#Unsicherheit Widerstand 1}
\end{Highlighting}
\end{Shaded}

\begin{verbatim}
## [1] 5863.556
\end{verbatim}

\begin{Shaded}
\begin{Highlighting}[]
\FunctionTok{sqrt}\NormalTok{(((}\DecValTok{1}\SpecialCharTok{/}\NormalTok{(}\FloatTok{101.8}\SpecialCharTok{*}\DecValTok{10}\SpecialCharTok{**{-}}\DecValTok{3}\NormalTok{))}\SpecialCharTok{*}\FloatTok{0.0029}\NormalTok{)}\SpecialCharTok{**}\DecValTok{2}\SpecialCharTok{+}\NormalTok{(}\FloatTok{0.4}\SpecialCharTok{/}\NormalTok{((}\FloatTok{101.8}\SpecialCharTok{*}\DecValTok{10}\SpecialCharTok{**{-}}\DecValTok{3}\NormalTok{)}\SpecialCharTok{**}\DecValTok{2}\NormalTok{)}\SpecialCharTok{*}\FloatTok{0.029}\SpecialCharTok{*}\DecValTok{10}\SpecialCharTok{**{-}}\DecValTok{3}\NormalTok{)}\SpecialCharTok{**}\DecValTok{2}\NormalTok{) }\CommentTok{\#Unsicherheit Widerstand 2}
\end{Highlighting}
\end{Shaded}

\begin{verbatim}
## [1] 0.02850921
\end{verbatim}

Die Widerstände für den Aufbau (b) ergeben sich somit insgesamt zu:

\begin{itemize}
\item $R1b = (1006000\pm 5900)\Omega $
\item $R2b = (3,929\pm 0,029) \Omega$
\end{itemize}

\end{document}

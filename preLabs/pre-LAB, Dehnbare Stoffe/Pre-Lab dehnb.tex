\documentclass[a4paper, 12pt]{article}

\usepackage[ngerman]{babel}
\usepackage[T1]{fontenc}
\usepackage{amsmath}

\usepackage[a4paper,
            bindingoffset=0.2in,
            left=1cm,
            right=1cm,
            top=1in,
            bottom=1in,
            footskip=.25in]{geometry}
            
\title{Pre-LAB: Dehnbare Stoffe, Team 4}
\author{Justus Weyers, Milena Mensching}

\begin{document}

\maketitle
\section{Hookesches Gesetz}
\textbf{Beschreiben Sie das Hookesches Model}

Das Hookesche Gesetz beschreibt die elastische Verformung dehnbarer Stoffe durch eine Kraft. Bei dem Körper/ Stoff handelt es sich beispielsweise um eine Feder. Die Spannkraft F ist dabei proportional zur Längenänderung $\Delta x$. Die Proportionalitätskonstante heißt Federkonstante $D$.

\begin{equation}
\begin{split}
\Rightarrow F&\sim\Delta x\\
 F&= -D*\Delta x
\end{split}
\end{equation}


\section{Annahmen beim Hookeschen Modell}
\textbf{Welche Annahmen werden beim Hookeschen Modell vorausgesetzt?}

\begin{itemize}
\item{Vernachlässigung von Energieumwandlung (z.B.: durch Reibung, $W=F_s*s$)}
\item{Lineare Kraft-Auslenkungs-Beziehung}
\item{Der Stoff muss dehnbar sein, die Elastizitätsgrenze darf jedoch nicht überschritten werden.}
\item{Keine plastische Verformung der Feder während mehrerer Messungen.}
\item{Gleiches Verhalten bei und Expansion und Relaxation der Feder}
\end{itemize}

\section{Experimentelle Ermittlung der Federkonstante}
\textbf{Wie kann man die Federkonstante einer idealen Feder durch ein Experiment ermitteln?}

Die Federkonstante einer idealen Feder kann durch einen Zugversuch ermittelt werden. Dabei werden mehrere (im besten Fall geeichte) Gewichte ($m_{i}$) an eine befestigte, senkrecht hängende Feder gehängt. Die Auslenkung $x_i$ gegenüber der Nullauslenkung $x_0$ durch die Zugkraft $F_i$ wird mit Hilfe einer geeigneten Skala gemessen.

%F_G &= F_k\\
%\vert m*g\vert &= \vert -D*\triangle \bar{x}\vert\\
%\Rightarrow D &= \frac{m*g}{k*\triangle \bar{x}}

Mit Gleichung 1 ergibt sich für die Federkonstante $D$:
\begin{equation}
\begin{split}
D_i &= -\frac{F_i}{\Delta x_i}\\
\Rightarrow  &=-\frac{g*m_i}{x_i-x_0}\\
\end{split}
\end{equation}

\noindent $g$: Erdbeschleunigung $9,81\frac{m}{s^2}$

Mit den $i$ bestimmten Werten für $D$ kann, gemäß einer Typ-A Messunsicherheit nach GUM, der Mittelwert, die Standardabweichung und die Standardabweichung des Mittelwertes berechnet werden. 

\newpage
\section{Messunsicherheiten}
\textbf{Wie wird die dazugehörige Messunsicherheit (der Federkonstante) berechnet?}
\begin{table}[h]
\centering
\begin{tabular}{|l|l|}
\hline
\textbf{Mehrere Messungen} & \textbf{Einmalige Messung}\\
\hline
Standardabweichung des Mittelwerts ($\sigma_{\bar{x}}$) & Ablesefehler ($u_{skala}$)\\
evtl. Unsicherheit aus Gewichtsmessung ($u_G$)& evtl. Unsicherheit aus Gewichtsmessung($u_G$)\\
\hline
\multicolumn{1}{|l}{\textbf{Formeln:}}& \\
\hline
$u_G$ abhängig von Art der Messung & $u_G$ abhängig von Art der Messung\\
\hline
Mittelwert: $\bar{x} = \frac {1}{n}\sum_{i}^n x_i$ &  $u_{skala} = \frac {a}{2\sqrt{6}}$\\
\hline
Standardabweichung: $\sigma = \sqrt{\frac{1}{n-1}\sum_{i}^n (x_i - \bar{x})^2}$ &  \\
\hline
Standardabweichung des Mittelwerts: $\sigma_{\bar{x}}= \frac{\sigma}{\sqrt{n}}$ & \\
\hline
\end{tabular}
\end{table}

\section{Gummiband Kraftsensor}
\textbf{Wie könnte man mit einem Gummiband ein Kraftsensor bauen?}

Der Prozess der Ausdehnung eines Gummibandes lässt sich nicht unbedingt (oder nur für bestimmte Dehnungsgrade) durch das Hooksche Gesetz beschreiben.
Die Auslenkungs-Kraft-Beziehung ist anders als zum Beispiel bei einer Feder nicht linear, es lässt sich also keine Proportionalitätskonstante (Federkonstante) festlegen.

Für einen Kraftsensor, welcher ein Gummiband verwendet, muss die Auslenkungs-Kraft-Beziehung beispielsweise als Schaubild, idealerweise als mathematische Funktion, bekannt sein. 
Dann ließe sich dieses Gerät analog zu einem Messgerät mit Feder verwenden. Zum Bestimmen der Kraft müsste dann auf dem Schaubild die der Auslenkung entsprechende Zugkraft abgelesen werden.

Typischerweise beruht allerdings bei Kraftsensoren oder auch bei Waagen die Messung auf der Verwendung von Federn. Dies ist praktischer als die Verwendung eines Gummibandes und durch das Hooksche Gesetz beschreibbar.
\end{document}